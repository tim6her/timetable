\documentclass[12pt,a4paper, landscape]{article}
\usepackage[applemac]{inputenc}
\usepackage[left=2cm,right=2cm,top=2cm,bottom=2cm]{geometry}
\usepackage[german]{babel}
\usepackage[T1]{fontenc}
\usepackage{amssymb}
\author{Tim Herbstrith}
\pagestyle{empty}

\usepackage{tikz}
\usepackage{pgfcore}
\usetikzlibrary{shapes.multipart}

\usepackage{array}
\newcolumntype{L}[1]{>{\raggedright\let\newline\\\arraybackslash\hspace{0pt}}m{#1}}
\newcolumntype{C}[1]{>{\centering\let\newline\\\arraybackslash\hspace{0pt}}m{#1}}
\newcolumntype{R}[1]{>{\raggedleft\let\newline\\\arraybackslash\hspace{0pt}}m{#1}}

\usepackage{setspace}
\renewcommand{\arraystretch}{1.5}

\newcommand{\firstH}{11}
\newcommand{\lastH}{19}
% Automatically draws Calendar Entries
\newcommand{\appointment}[5]{
\filldraw[black!20, rounded corners] (#1-0.99,\firstH -#2) rectangle (#1-0.01,\firstH -#3);
\path (#1-1,\firstH -#2) -- (#1,\firstH -#3) node[circle split,midway, font=\scriptsize, text width=3cm, minimum width=1, text centered] {#4 \nodepart{lower} #5};
}

\begin{document}
\begin{center}

\begin{tikzpicture}[xscale=4, yscale=1]
\foreach \hour in {\firstH,...,\lastH}
	\draw (0,\firstH-\hour) node[left=5] {\hour.00} -- (5,\firstH-\hour)  (0,\firstH+0.5-\hour) -- (5,\firstH+0.5-\hour);
\foreach \day in {0,...,5}
	\draw[line width=8pt,white] (\day,0.6) -- (\day,\firstH-\lastH -0.1);
\foreach \day/\name in {1/Montag,2/Dienstag,3/Mittwoch,4/Donnerstag,5/Freitag}
	\node at (-0.5+\day,1) {\name};
	
\appointment{1}{15.00}{16.50}{Jour Fixe}{01.127}
\appointment{2}{11.25}{12.83}{Stochastische Analysis}{SR 11}
\appointment{3}{15.00}{16.50}{Hilfsmittel aus der EDV}{PC 02}
\appointment{3}{17.25}{18.75}{Fragestunden}{HS 13}
\appointment{4}{17.25}{18.75}{Fragestunden}{HS 13}
\appointment{4}{11.25}{12.83}{Stochastische Analysis}{SR 08}
\end{tikzpicture}

\vspace*{\fill}
\begin{tabular}{L{25em} C{4em} C{25em} C{2em}}
\textbf{Vorlesung zu	}		&	\textbf{ECTS}	&	\textbf{Termin}\\
\hline
\hline
Einf\"uhrung in die Analysis	&	5	&	&	\\
Analysis						&	7	&	&	\\
Numerische Mathematik		&	3	&	5. Dezember 2014		&	\\
\hline
Diskrete Mathematik			&	3	&		&	\\
Funktionalanalysis			&	5	&	
\end{tabular}
\end{center}
\end{document}